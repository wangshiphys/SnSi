\documentclass[aps,prb,reprint,amsfonts,amsmath,amssymb,showpacs,groupedaddress,superscriptaddress,onecolumn]{revtex4-1}
\usepackage{bm}
\usepackage{color}
\usepackage{graphicx}
\usepackage{tabularx}
\usepackage[colorlinks,urlcolor=blue,linkcolor=blue,anchorcolor=blue,citecolor=blue,bookmarks]{hyperref}

\begin{document}

\title{Artificial two-dimensional Mott insulating superstructures with a large Mott gap: Theoretical Formalism}

\date{\today}

\maketitle

To have a better understading of the experimental observations, we construct single-band Hubbard model with renormalized hopping coefficients to describe these systems and calculate the local density of states by using cluster perturbation theory (CPT)~\cite{PhysRevB.48.418,PhysRevLett.84.522}.

For the $(3\sqrt{7} \times 3\sqrt{7})R19.1^\circ$ surface (referred to as Phase 1 in the followings), the corresponding proposed atomic structural is shown in Fig.~\ref{fig:STMTopographicImage}(g). The unit cell contains twenty-three sites while three of them are somewhat isolated, i.e., the three blue circles are speparated from the others. Considering that the hopping amplitude is inversely proportional to the square of the distance, it is reasonable to neglect the three isolated sites in a simplified model. We only include these hopping terms shown in Fig.~\ref{fig:ModelForPhase1} and the resulting Hamiltonian takes the following form:
\begin{equation}
    H = \sum_{i,j,\sigma} t_{ij}(c_{i\sigma}^{\dagger}c_{j\sigma} + \text{H.c.}) + U \sum_{i} n_{i\uparrow} n_{i\downarrow}
    \label{eq:ModelHamiltonian}
\end{equation}
where $t_{ij}$ is the effective hopping amplitude and $U$ the effective on-site Coulomb repulsion. Here, we take $t_{ij} = -1/r_{ij}^{2}$, where $r_{ij}$ is the distance between $i$-th site and $j$-site. See Appendix~\ref{appx:Coordinates} for the coordinates of these sites shown in Fig.~\ref{fig:ModelForPhase1}.

For this model Hamiltonian, we first consider the non-interacting case (\textit{i.e.,} $U = 0$) which can be diagonalized exactly. The averaged DOS and representative LDOS are shown in Fig.~\ref{fig:TBAForPhase1}. It can be clearly seen from the averaged DOS that the system has an energy gap of $\sim 0.55t$. However, the LDOS varies from site to site. For instance, the LDOS of the site labeled as ``0'' is quite different from the sites labeled as ``1'', ``2'' and ``3''. Especially, the energy gap at the zeroth site is $\sim 3.2t$, and yet that at other sites is $\sim 0.55t$. This inhomogeneous LDOS is obviously contradictory to the experimental uniform insulating gap. We further include on-site Coulomb repulsion and use CPT to calculate the LDOS. The evolution of the averaged DOS with U is shown in Fig.~\ref{fig:CPTForPhase1}. The energy gap becomes gradually decreased when taking into account of Hubbard-$U$ and a transition from band insulator to metal is identified at $U \approx 3.5t$. As we further increase $U$, an energy gap is reopened, indicating a metal-insulator transition. The including of $U = 6t$ leads to an experimentally comparable large energy gap of $\sim 2.31\text{eV}$. We further explored the corresponding LDOS and found that the energy gap is uniform at each lattice sites from 0 to 19 (see Fig.~\ref{fig:CPTForPhase1LDOS}), which is consistent with the experimentally observed homengeneous dI/dV spectra. The agreement between the experimental observation and theoretical calculation indicates the Mott origin of the large energy gap of $\sim 2.5\text{eV}$ for $(3\sqrt{7} \times 3\sqrt{7})R19.1^\circ$ surface.

For $(\sqrt{133} \times 4\sqrt{3})$ and $(13 \times 13)$ surfaces (referred to as Phase 2 and Phase 3 respectively), we similarly constructed the simplified model for theoretical calculations.


\begin{figure}[p]
    \includegraphics[width=0.8\columnwidth]{STMTopographicImage.png}
    \caption{\label{fig:STMTopographicImage}(Color online) STM characterization of three new superstructures of Sn sub-monolayers on Si(111). (a)-(c) Large-scale STM image (size: 100 $\times$ 100 nm$^2$) taken on $(3\sqrt{7} \times 3\sqrt{7})R19.1^\circ$, $(\sqrt{133} \times 4\sqrt{3})$ and $(13 \times 13)$ surfaces. They are taken at $U = +3.5V$, $U = -2V$ and $U = -2V$ ($I_{t} = 100pA$) respectively. (d)-(f) The atomically resolved STM images of them taken at $U = -2V, I_{t} = 200pA$. The surface unit cells of them are marked in black. (g)-(i) The corresponding proposed atomic structural models. The marked surface unit cells in (g)-(i) are the same as these in (d)-(f).}
\end{figure}

\begin{figure}[p]
    \includegraphics[width=0.8\columnwidth]{ModelForPhase1.pdf}
    \caption{\label{fig:ModelForPhase1} (Color online) Demonstration of the hopping terms for Phase 1. The unit cell has twenty sites labeled from 0 to 19. The soild and dashed lines correspond to the hopping terms $t_{ij} (c_{i\sigma}^{\dagger} c_{j\sigma} + \text{H.c.})$, where $t_{ij} = -1/r_{ij}^{2}$ and $r_{ij}$ is the distance from $i$-th to $j$-th site.}
\end{figure}

\begin{figure}[p]
    \includegraphics[width=0.8\columnwidth]{TBAForPhase1.pdf}
    \caption{\label{fig:TBAForPhase1} (Color online) Local density of states calculated from the tight-binding model defined in Fig.~\ref{fig:ModelForPhase1}. The blue line is the average of the LDOS over the twenty sites in a unit cell. LDOS for site \{0, 10\}, \{1, 4, 7, 12, 15, 18\}, \{2, 5, 8, 13, 16, 19\}, \{3, 6, 9, 11, 14, 17\} are the same. The gray dashed vertical line marks the Fermi energy $E_{F}$.}
\end{figure}

\begin{figure}[p]
    \includegraphics[width=0.8\columnwidth]{CPTForPhase1.pdf}
    \caption{\label{fig:CPTForPhase1} (Color online) The evolution of averaged DOS with Hubbard-$U$.}
\end{figure}

\begin{figure}[p]
    \centering
    \includegraphics[width=0.8\columnwidth]{CPTForPhase1LDOS0.pdf}
    \includegraphics[width=0.8\columnwidth]{CPTForPhase1LDOS1.pdf}
    \caption{\label{fig:CPTForPhase1LDOS} Local density of states at $U = 6t$.}
\end{figure}

\appendix

\section{\label{appx:Coordinates}Coordinates of these sites in Phase 1}

\begin{table}[h]
    \centering
    \begin{tabular}{c | l | c | l}
        \hline
        index & (x, y) & index & (x, y) \\
        \hline
        0 & (0, 0) & 1 & (-1.5, $\sqrt{3} / 6$) \\
        \hline
        2 & (-1.0, -$\sqrt{3}$ / 3) & 3 & (-0.5, -5$\sqrt{3}$ / 6) \\
        \hline
        4 & (0.5, -5$\sqrt{3}$ / 6) & 5 & (1.0, -$\sqrt{3}$ / 3) \\
        \hline
        6 & (1.5, $\sqrt{3}$ / 6) & 7 & (1.0, 2$\sqrt{3}$ / 3) \\
        \hline
        8 & (0.0, 2$\sqrt{3}$ / 3) & 9 & -1.0, 2$\sqrt{3}$ / 3 \\
        \hline
        10 & (4.5, $\sqrt{3}$ / 2) & 11 & (3.0, $\sqrt{3}$ / 3) \\
        \hline
        12 & (3.5, -$\sqrt{3}$ / 6) & 13 & (4.5, -$\sqrt{3}$ / 6) \\
        \hline
        14 & (5.5, -$\sqrt{3}$ / 6) & 15 & (6.0, $\sqrt{3}$ / 3) \\
        \hline
        16 & (5.5, 5$\sqrt{3}$ / 6) & 17 & (5.0, 4$\sqrt{3}$ / 3) \\
        \hline
        18 & (4.0, 4$\sqrt{3}$ / 3) & 19 & (3.5, 5$\sqrt{3}$ / 6) \\
        \hline
        20 & (8.0, $\sqrt{3}$ / 3) & 21 & (10.5, 5$\sqrt{3}$ / 6) \\
        \hline
        22 & (8.5, 11$\sqrt{3}$ / 6) & & \\
        \hline
    \end{tabular}
    \caption{\label{tab:Coordinates}Coordinates of these sites in Phase 1.}
\end{table}


\bibliography{TheoreticalFormalism}

\end{document}
